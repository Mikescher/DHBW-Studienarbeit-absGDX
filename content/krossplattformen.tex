Um ein Spiel oder eine App nicht nur für ein Betriebsystem (Plattform) zum entwickeln, sondern dieses Spiel so weit
wie möglich zu fächern, gibt es die Möglichtkeit der Krossplattformen.
Dadurch lässt sich ein Spiel oder eine App für mehrere Betriebsysteme wie zum Beispiel Android, iOS oder Windows Phone
entwickeln. Dabei können die Anwendungen auch für anderen Plattformen
als den Mobilen entwickelt werden, hierzu zählen unter anderem Anwendungen für den Desktop PC 
oder auch für HTML5. Was für den Entwickler selbst eine große Hilfe ist, da er den Code nur einmal
schreiben muss.
Um für verschiedene Plattformen zu entwickeln gibt es mitlerweile schon eine Reihe von Frameworks,
die dies ermöglichen. Unter anderem Cocos2D, MonoGame, Unity und Libgdx, auf diese Frameworks wird im 
folgenden noch näher eingegangen und erläutert.