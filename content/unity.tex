Unity3D ist das wohl größte Framework, das wir uns angeschaut haben. Es ist nicht nur eine Ansammlung von Bibliotheken, sondern stellt auch eine Vielzahl an Werkzeugen und Editoren zur Verfügung. Der zentrale Code wird in \CS geschrieben. Jedoch fließen weitere Skriptsprachen wie Boo oder UnityScript in die Projekte mit ein.
Zusätzlich zu einer speziell für Unity3D entwickelten IDE, gibt es auch Editoren für die verschiedenen Modellformate, die Unity3D benutzt.

Unity3D ist - wie der Name es schon sagt - primär eine dreidimensionale Spieleengine, kann jedoch auch auf zwei Dimensionen reduziert werden.
Unity3D unterstützt eine Vielzahl an Zielplattformen. Nicht nur Windows, Mac, Linux, Android und iOS sondern auch Konsolen wie die Playstation 3, Playstation 4, XBox360 und die Wii. Zwar ist es nicht möglich, die Spiele nach Javascript zu kompilieren jedoch könne sie über den Unity-WebPlayer, ähnlich wie Flash Inhalte in eine HTML Seite, eingebunden werden.