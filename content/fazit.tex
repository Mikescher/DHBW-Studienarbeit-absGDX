Als Ergebnis unserer zwei Semester dauernden Entwicklungsarbeit haben wir es tatsächlich geschafft ein 2D-Framework auf die Beine zu stellen.
Mehrere größere und kleiner Beispielprogramme haben uns gezeigt, dass wir unser Ziel erreicht haben es zu ermöglichen leichter 2D Apps zu erstellen als mit LibGDX direkt.
Auch haben wir vor das Framework in Zukunft noch weiter zu entwickeln und für andere Projekte zu verwenden.

Das wohl interessanteste waren die Vorteile, welche wir durch Unit-Tests und Continuous Integration bekommen haben. In mehreren Fällen haben wir dadurch Commits identifiziert, welche Bugs eingeführt haben und ebenso haben wir Probleme teilweise schon erkannt bevor sie wirklich aufgetreten sind.

Das größte Problem was unser Framework aktuell hat ist die Performance auf Low bis Mid-tier Android Geräten. Während ein mittelmäßiger PC auch eine große Karte mit vielen Entities rendern kann haben mobile Geräte hier viel schneller Probleme. Bei unserer Weiterentwicklung sollten wir uns also zuerst einmal auf Optimierungen konzentrieren. Trotzdem ist es klar, dass man auf einem Smartphone wohl nie die gleiche Performance wie auf einem ganzen PC haben wird.

Auch fehlen noch einige Features, für welche wir vor der Abgabe keine Zeit mehr hatten, die wir aber gerne noch sehen würden. Darunter ein Partikelsystem, eine Möglichkeit Menüelemente über dem Spiel anzuzeigen (Stichwort HUD) und ein Netzwerksupport. Gerade aufgrund dieser fehlenden Funktionen und der Tatsache, dass wir zuerst einmal selbst ein größeres Projekt damit entwickeln wollen (um eventuell noch weitere Schwachstellen des Frameworks aufzudecken) haben wir uns zuerst einmal dagegen entschieden das Projekt richtig publik zu machen. 
Sobald wir jedoch einen Stand erreicht haben mit dem wir zufrieden sind, haben wir vor das Framework auf verschiedenen Foren vorzustellen und vielleicht so auch ein paar Programmierer dazu zu bewegen es zu benutzen.

Denn der Vorteil, den wir am Anfang in unserem Framework gesehen haben ist immer noch relevant: Es ist extrem komfortabel ein Spiel an einem Desktop Rechner zu entwickeln und zu testen ohne es auf ein Smartphone kopieren zu müssen. Dies haben auch unsere eigenen Erfahrungen bestätigt. 
%TODO Armin? wilst du auch noch was sagen - kA was in so ein Fazit rein soll