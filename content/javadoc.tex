% API Documentation

Da unser Ziel es ist ein Framework zu entwickeln, dass auch von anderen Entwicklern genutzt wird dokumentieren wir alle öffentlichen Methoden. Mithilfe des Dokumentationswerkzeugs \ti{javadoc} kann man aus diesen Quellcode Kommentaren HTML Dokumentationsseiten erzeugen. Im nachfolgenden möchte ich die verschiedenen Schritte und Werkzeuge aufzeigen welche wir benutzen um eine Dokumentationsseite automatisch online zu hosten.

Die eigentlichen Dokumentationskommentare sind als Kommentare in den Quellcodedateien vorhanden und werden somit auch in unser git Repository commited. Unsere IDE Eclipse hilft uns dabei die Kommentare syntaktisch korrekt nach den javadoc Vorgaben zu schreiben, so dass diese später geparsed werden können.

Das Erzeugen der entsprechenden HTML, CSS, und JS Dateien wird komfortabler Weise von gradle übernommen. Im gradle Plugin \ti{java} gibt es den Task \ti{javadoc} und mit dem folgenden Befehl kann man sich einfach eine javadoc Webseite generieren lassen: 

\doinlineII
\begin{lstlisting}[caption=Gradle in der Kommandozeile, title=\hspace{0 pt}, style=cmd]
> gradlew absGDX-framework:javadoc
\end{lstlisting}

Die nächste Frage ist wo man das javadoc hosten soll. Hierfür gibt es bei unserem git Hoster GitHub die sogenannten \ti{GitHub-pages}. Mit ihnen kann man eine beliebige HTML Seite für jede Repository anlegen. Dafür pushed man die entsprechenden Dateien in ein eigenen Branch \textbf{gh-pages}. Unter der URL \ti{http://USERNAME.github.io/REPOSITORY/} kann man dann auf diese Webseite zugreifen. In unserem Fall kann man unter \href{http://mikescher.github.io/absGDX/javadoc/}{http://mikescher.github.io/absGDX/javadoc/} online die Dokumentation einsehen.

Trotzdem ist es müßig nach jeder Änderung die Dokumentation neu zu erstellen und hochzuladen. Dieser Prozess wartet nur darauf, dass man dies irgendwann einmal vergisst.
Wir können jedoch unser Continuous Integration Skript verwenden um nicht nur das Projekt zu überprüfen sondern auch um automatisiert bei jedem \ti{push} auf das Repository die javadoc Dateien zu aktualisieren.
Hierfür legen wir in dem Ordner ".utility" das Bash Skript "push-javadoc-to-gh-pages.sh" an. In unserer ".travis.yml" Konfigurationsdatei müssen wir dann nach einem erfolgreichen Test zuerst das javadoc erzeugen und dann die Dateien per git commiten. Ein Problem ist dabei, dass wir vom TravisCI Server aus keine Rechte haben auf unser Repository zu pushen. Deshalb generieren wir für unser Repository ein Access Token und geben dieses verschlüsselt in unserem Skript an. Dies ist möglich da TravisCI automatisch für jede Repository ein RSA Schlüsselpaar generiert. Unser Token haben wir nun mit dem Public Key verschlüsselt und nur Travis kann ihn mit dem Private Key wieder entschlüsseln. Somit sind wir sicher davor, dass sich jemand anders die \ti{(öffentliche)} ".travis.yml" Datei ansieht und den Access Token missbraucht.

\doinline
\begin{lstlisting}[caption=TravisCI Konfiguration für das Erzeugen von JavaDoc, title=\hspace{0 pt}, language=groovy]
language: java
env:
  global:
   - secure: "?? encrypted ??"

before_install:
 - chmod +x gradlew
 - chmod +x .utility/push-javadoc-to-gh-pages.sh

after_success:
 - ./gradlew absGDX-framework:javadoc
 - .utility/push-javadoc-to-gh-pages.sh
\end{lstlisting}

In unserem eigentlichen Skript überprüfen wir zuerst einmal ob wir auch wirklich die aktuelle Dokumentation commiten wollen. Zuerst muss der Commit auf dem richtigen Repository erfolgt sein, zur Sicherheit damit niemand aus versehen das Skript forked und dann auf seiner Repository ausführt. Außerdem muss es ein Commit auf dem Branch \ti{master} sein und ein echter push \ti{(nicht nur ein Pull Request)}.

Danach kopieren wir die generierten Dateien in das Verzeichnis "tmp\_jd" und klonen mit dem imaginären Benutzer "travis-ci" und unserem Access Token den aktuellen Zustand der Branch gh-pages. 
Die gerade geklonten Dateien löschen wir jedoch sofort wieder und ersetzen sie mit den Neuen. 
Für unveränderte Dateien ist das effektiv eine No-Operation, nur geänderte Dateien wurden mit den neueren Versionen ersetzt. 
Diese geänderten Exemplare werden dann mit \textbf{"git add"} zum index hinzugefügt. 
Danach werden sie mit \textbf{"git commit"} committed und mit \textbf{"git push"} zum Remote (Github) gepushed.

\doinline
\begin{lstlisting}[caption=Das Javadoc Publish Bashscript, title=\hspace{0 pt}, language=bash]
if [ "$TRAVIS_REPO_SLUG" == "Mikescher/absGDX" ] && [ "$TRAVIS_PULL_REQUEST" == "false" ] && [ "$TRAVIS_BRANCH" == "master" ]; then
	cp -R absGDX-framework/build/docs/javadoc $HOME/tmp_jd

	cd $HOME

	git config --global user.email "travis@travis-ci.org"
	git config --global user.name "travis-ci"
	git clone --branch=gh-pages https://${GH_TOKEN}@github.com/Mikescher/absGDX gh-pages
	cd gh-pages

	git rm -rf ./javadoc

	cp -Rf $HOME/tmp_jd ./javadoc

	git add -f .
	git status
	git commit -m "Lastest javadoc on successful travis build $TRAVIS_BUILD_NUMBER auto-pushed to gh-pages"
	git push -f origin gh-pages
fi
\end{lstlisting}

