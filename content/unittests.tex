% Problem: Android duplicate files when packaging (hamcrest <> junit)
% => Solution: New Test Project
% Parametrized tests
% jUnit 4
% Vector compare

Um die Codequalität zu erhöhen haben wir uns entschieden unseren Code mit Unittests abzusichern.
Die Unittests schreiben wir mit Test-Framework jUnit 4 \cite{JUNIT}. Dies hat den Vorteil, dass es schon fertige Eclipse Plugins gibt um die Tests auszuführen und die Ergebnisse zu visualisieren.

\quickfigure{Beispiel der jUnit Test Anzeige}{junit-overview}{0.6}

Das Ziel der Unittests ist es einen Großteil der Funktionen und Methoden des Frameworks zu testen in dem sie mit verschiedenen Parametern getestet werden und dann überprüft wird ob die Ergebnisse stimmen.
Besonders bei der Kollisionserkennung ist dies nützlich, da wir so überprüfen können ob nach Änderungen an der \keyword{Collisionmap} beispielsweise noch alle Kollisionen erkannt werden. Und es gibt sehr viele verschiedene Fälle von Kollisionen die überprüft werden müssen.

\paragraph{Eigene Asserts}

Da das Ergebnis meist in Form eines \keyword{Vector2} vorliegt haben wir eine Assert Methode geschrieben, welche überprüft ob zwei Vektoren gleich sind

\doinline
\begin{lstlisting}[caption=Eine Assert Methode für zwe Vektoren, title=\hspace{0 pt}, language=java]
public static void assertEqualsExt(Rectangle expected, Rectangle actual, float epsilon) {
	boolean a = !fcomp(expected.x, actual.x, epsilon);
	boolean b = !fcomp(expected.y, actual.y, epsilon);
	boolean c = !fcomp(expected.width, actual.width, epsilon);
	boolean d = !fcomp(expected.height, actual.height, epsilon);
	
	if (a || b || c || d) {
		throw new AssertionFailedError("expected:<" + expected + "> but was:<" + actual + ">" );
	}
}

private static boolean fcomp(float expected, float actual, float delta) {
	return Float.compare(expected, actual) == 0 || (Math.abs(expected - actual) <= delta);
}
\end{lstlisting}

Da wir hier Gleitkommazahlen vergleichen, welche eine endliche Genauigkeit haben, muss ein epsilon Wert eingeführt werden. Dieser ist die maximale Abweichung die zwei \ti{floats} haben dürfen um noch als gleich angesehen zu werden.

\paragraph{Parametrisierte Tests}

Ein Feature von jUnit 4, dass wir benutzen sind die parametrisierten Tests\cite{JUPT}. 
Hierbei führt man einen Test mehrmals unter verschiedenen Voraussetzungen aus. 
Wir benutzen dies intensiv bei den Tests der Kollisionserkennung. 
Jeder Kollisionstest wird mit mehreren verschiedenen Collisionmap-Größen ausgeführt. 
Einmal mit normal großer Map, einmal mit einer Map der Größe 1x1 und auch noch mit verschiedenen anderen Größen. 
Dies erhöht zwar die Anzahl der auszuführenden Tests stark, sorgt jedoch dafür, dass es wahrscheinlicher wird Fehler zu erkennen. 
Und bei Unittests ist die Performance in soweit nicht sehr wichtig da sie nur ab und zu auf dem Entwicklerrechner oder dem CI-Server ausgeführt werden.

\paragraph{Android Build Probleme}

Ein Problem, dass wir mit jUnit jedoch hatten war der Android Build Vorgang. 
Anfangs waren all unsere Unittests in dem \keyword{absGDX-framework} Projekt. Deshalb war dieses Projekt von jUnit abhängig und jUnit ist seinerseits von der Hamcrest Bibliothek abhängig\cite{HCRST}. 
Das Android Projekt \keyword{android}, dass wir zum Kompilieren für Android verwenden ist von den Android Libraries und \keyword{absGDX-framework} abhängig.
Problematisch wird es hier jedoch da auch die Standard Android Libraries Hamcrest enthalten und Hamcrest im root Ordner der jar/apk eine Datei \ti{LICENSE.txt} anlegen will. Der Android Compiler verbietet es jedoch, dass zwei Libraries die gleiche Datei anlegen wollen (auch wenn es in diesem Fall zweimal die gleiche Bibliothek ist).
Gradle hat hierfür eine Lösung. Mit der Einstellung packagingOptions.exclude kann man Dateien aus dem Paket ausschließen. Der Eclipse Compiler ist jedoch nicht so intelligent und somit führt das alles dazu dass wir nur noch mit gradle das Android projekt bauen könne und nicht mehr direkt in Eclipse.

Die Lösung hierfür bestand darin die Unittests in ein eigenes Projekt \keyword{absGDX-test} auszulagern welches von \keyword{absGDX-framework} abhängig ist. Somit ist \keyword{absGDX-framework} selbst nicht mehr von jUnit und hamcrest abhängig und das Projekt \keyword{android} bekommt die hamcrest-Bibliothek nur noch einmal eingebunden.