Es ist oftmals nötig ein Programm zu debuggen. Hierfür gibt es zwei Ansätze, einerseits kann man das Programm mit einem Debugger an einer bestimmten Stelle anhalten, den Speicher auslesen und den Code Zeile für Zeile durchlaufen.
Dies ist nützlich bei Programmen welche etwas berechnen. Da wir hier jedoch ein fortwährend laufendes Programm haben ist es manchmal nötig das Programm \ti{live} zu beobachten. Hierfür haben wir den \keyword{DebugTextRenderer}. Mit ihm kann man, ohne das Programm anzuhalten, im Bildschirm Debuginformationen anzeigen lassen.

\quickfigure{Beispiel der Debugansicht (In-Game)}{debugview_game}{0.95}

Außerdem können die Abgrenzungen von Entities, Tiles und Kollisionsgeometrien angezeigt werden. Zusätzlich werden die aktuellen physikalischen Eingenschaften (Position, Geschwindigkeit, Beschleunigung) von Entities mittels Pfeilen visualisiert.

Ähnliche Anzeigen sind auch im \keyword{MenuLayer} für \keyword{MenuElements} vorhanden und können wenn nötig vom Benutzer um eigene erweitert werden.

\quickfigure{Beispiel der Debugansicht (In-Menu)}{debugview_menu}{0.95}

Die Anzeige der Debugansicht wird über die \keyword{DependentProperties} geregelt. Die Root Property \ti{debugEnabled} regelt ob die Debugansicht aktiv ist. Mit eventuellen Unterproperties kann man verschiedene Ansichten aktivieren oder deaktivieren. Es ist nützlich das Ein/Ausschalten der Debugansicht auf eine Tastaturtaste oder einen Menüknopf zu legen (aufrufen der \ti{doSwitch()} Methode).