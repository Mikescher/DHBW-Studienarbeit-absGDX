Als letztes schauen wir uns LibGDX an. LibGDX ist ein OpenSource Java-Framework. Es ist primär auf Android, PC und Browser ausgelegt obwohl es auch Möglichkeiten gibt den Code nach iOS zu portieren. LibGDX ist im Vergleich zu den bisher vorgestellten Framework eher leicht gehalten, obwohl es eine Reihe an Bibliotheken gibt um seine Funktionen zu erweitern.

Wir haben uns für LibGDX entschieden weil es all unseren Anforderungen gerecht wurde, man kann Code (in Java) schreiben und sowohl für die DalvikVM als auch für die JVM kompilieren. LibGDX ermöglicht es mittels eines eigenen Wrappers plattformunabhängig auf OpenGL zuzugreifen, Trotzdem haben wir noch relativ direkten Zugriff auf die OpenGL API und können unser eigenes Rendering betreiben. Dies sorgt auch dafür, dass es leicht für uns ist OpenGL als zweidimensionalen Renderer zu benutzen.
Da LibGDX Gradle als Build Tool verwendet ist es einfach LibGDX als Dependency in unserem eigenen Projekt einzubinden und somit stellt es kein Problem dar ein eigenes Framework zu bauen welches LibGDX als Grundlage für seine Plattformunabhängigkeit nimmt.

Es ist noch zu erwähnen, dass LibGDX durchaus ein paar allgemein gehaltene Funktionen besitzt um die Spiele Entwicklung zu vereinfachen. Diese Benutzen wir jedoch nicht, da dass Ziel mit unserem Framework ist, einen viel spezielleren Fall (2D Tiled Games) abzudecken.