Zuletzt kommen wir zu der Plattform mit welcher in unserer Studienarbeit gearbeitet wird.
Libgdx ist eine Spieleentwicklungs Framwork die mit Java geschrieben wurde. Dabei sind einige C++ Elemente
vorhanden die f�r die Performens zust�ndig sind. Diese Multiplattform unterst�tzt Windows, Linux, Mac OS X,
Android, iOS, BlackBerry und Internetbrowsers mit WebGL(blablabla fusszeile) support. 
Dabei kann mit Libgdx noch weitere Add Ons genutzt werden, unter anderem Spine(2D Skeletal Animation),
Nextpeer(Mobile multiplayer made easy), Saikoa(makers of ProGuard and DexGuard)

Audio wird unterst�tzt durch abspielen von WAV, MP3, OGG. Eingaben von Maus, Touchscren, Tastatur,
Beschleunigungsensor wie auch der Kompass werden unterst�tzt, gestigen werden ebenfalls gehandhabt.
Auch Libgdx unterst�tzt ein Mathematik und Physik System. Auch kann man mit Libgdx auf die Dateisystem der
verschiedenen Plafftormen zugreifen.
Libgdx bringt auch eine vielzahl grafik Benutzungen mit. Zum Beispiel durch das Rendern mit OpenGL.
Und auch einige Hilfsklassen f�r Low Level OpenGL. Desweitern verf�gt Libgdx auch mehrere High-Level
2D APIs und auch 3D APIs.

Es werden einige Tools mitgeliefert. Unter anderem eine Partikel Editor, Texture packer und einen Bitmap
font Generator.