Cocos2D-x ist ein Open-Source C++ Framework. Dies ist unter Android ein besonderer Vorteil weil die DalvikVM umgangen wird und nativer Code auf dem Gerät ausgeführt wird.
Man verliert dadurch jedoch einige Features der VM wie Garbage Collecting, gewinnt aber unter Umständen an Performance weil eine Zwischenschicht wegfällt.
Cocos2D-x kann neben Windows und Android auch nach iOS kompilieren und auch mittels Javascript und HTML5 Browser spiele erzeugen.

Neben der Fähigkeit Programme für verschiedene Systeme zu erzeugen kommt Cocos2D-x auch mit einer Vielzahl an Bibliotheken und Werkzeugen um Spiele zu entwickeln an.

Wir haben uns dagegen entschieden Cocos2D zu verwenden, da der native Code zwar Vorteile hat, eines unserer Ziele es jedoch war das Entwickeln von Spielen für den Entwickler zu vereinfachen, und dies geht mit einer interpretierten Sprache einfacher. Außerdem kommt Cocos2D schon mit einer so gut ausgebauten Sammlung an Werkzeugen für zweidimensionale Spiele, dass wir entweder nur sehr wenig selber zu programmieren hätten oder an vielen Stellen schon bestehende Funktionalitäten neu Schreiben müssten.