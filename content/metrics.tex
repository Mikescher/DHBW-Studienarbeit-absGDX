\paragraph{Code Metrics}

Um eine übersicht über den aktuellen Stand unseres Codes zu erhalten setzten wir das Metrics Tool \textbf{Metrics 1.3.9} ein. 

\quickfigure{absGDX Metrics}{metrics}{1.0} %TODO UPDATE ME

Es zeigt eine Reihe an Statistiken über den Quellcode an und warnt wenn Teile des Codes komplex werden. 
Die Warnungen sind jedoch nicht immer korrekt, in unserem Beispiel sind drei Methoden mit Problemen markiert, wir haben uns jedoch bei allen drei entschieden sie so zu lassen. Zwar sehen sie rein statistisch sehr komplex aus, jedoch sind sie aus einer menschlichen Sichtweise sehr gut zu lesen.

Trotzdem sind die Metriken nützlich um schnell und einfach einen überblick über den aktuellen Stand zu bekommen und eventuelle Problemstellen zu identifizieren

\paragraph{Test Coverage}

Da wir Unit Tests mit jUnit einsetzen, ist eine interessante Frage wie viele Statements von den Unit Tests abgedeckt werden.
Das Code Coverage Tool \textbf{EclEmma} analysiert hierfür unsere Tests und zeigt - nach Paket sortiert - jeweils an wie viele Statements abgedeckt sind

\quickfigure{absGDX Test Coverage}{coverage}{1.0} %TODO UPDATE ME

Wie hier zu sehen ist haben wir keine hundert prozentige Testabdeckung. Trotzdem sind wichtige und fehleranfällige Pakete wie \ti{collisiondetection}, \ti{math} oder \ti{mapscaleresolver} fast vollständig abgedeckt.
Für die Zukunft kann es durchaus von Vorteil sein noch mehr des Quellcodes mit Tests abzudecken. Jedoch sind vorerst die wichtigsten Methoden schon abgedeckt.