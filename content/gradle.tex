%TODO Buch �ber gradle (? -> PDF) f�r Referenzen

Zwar commiten wir unser Code mit git in unsere Repository, jedoch nicht alle Dateien. Es gibt eine Reihe an Dateien die nur tempor�r sind und beispielsweise bei jedem Kompiliervorgang neu erzeugt werden. Es w�re sinnlos solche Dateien in die Versionskontrolle aufzunehmen, vor allem da sie sich immer �ndern w�rden. Es gibt jedoch auch Dateien die zwar wichtig sind, trotzdem aber nicht in die Versionskontrolle geh�ren. Dies sind einerseits Konfigurationsdateien mit lokalen Pfaden die ung�ltig auf anderen System w�ren, IDE spezifische Dateien die unn�tz f�r Nutzer anderer IDE's w�ren oder externe Libraries die oftmals zu gro� sind um sie alle in die Repository zu commiten.

Um diese Probleme k�mmert sich nun ein weiteres Tool namens gradle. Gradle ist ein Build-Managment-Automatisierung-Tool was bedeutet, dass wir in unser Git-Repository nur noch die gradle Konfigurationsdateien commiten und diese dann jeweils auf den Entwicklerrechnern ausf�hren. Diese l�st dann die Projektabh�ngigkeiten auf und l�dt fehlende Libraries automatisch nach und erstellt au�erdem Dateien wie die IDE-Projektdateien automatisch.

\paragraph{Dependency Managment}

In gradle kann man angeben von welchen Bibliotheken oder Projekten ein Programm abh�ngig ist. Diese Bibliotheken k�nnen dann wiederum von anderen Bibliotheken abh�ngig sein und so bildet sich ein Abh�ngigkeitsbaum ausgehen vom Anfangsprojekt.
Gradle l�dt dann diese Bibliotheken, falls sie noch nicht auf dem Rechner vorhanden sind. Dies hat auch den Vorteil dass wenn eine Bibliothek �fters im Abh�ngigkeitsbaum auftaucht sie trotzdem nur einmal geladen werden muss.

Gradle unterst�tzt viele verschiedene Repositories wo nach fehlenden Bibliotheken gesucht werden kann, standardm��ig wird jedoch die \ti{Maven Central Repository} benutzt.
In dieser Repository sind sehr viele der frei verf�gbaren java Libraries in vielen Versionen enthalten und k�nnen einfach �ber zum Beispiel gradle geladen werden.

\paragraph{Konfiguration}

Die Konfiguration geschieht �ber \keyword{build.gradle} Dateien welche in \keyword{Groovy} geschrieben sind. 
absGDX ist ein Multiprojekt, das bedeutet es besteht aus mehreren Gradle Projekten mit jeweils eigenen Konfigurationen die voneinander abh�ngen.

\quickfigure{Gradle Dependency Graph absGDX}{dependency-graph-gradle.png}{1.0}

Das Projekt \keyword{absGDX-framework} ist hierbei das eigentliche Framework das wir entwickeln. \keyword{asGDX-test} ist das Testprojekt, es enth�lt alle Unit-tests f�r das Framework.
\keyword{absGDX-core} und die beiden Plattformprojekte \keyword{desktop} und \keyword{android} sind zum testen und debuggen. Wird das Framework sp�ter f�r ein Projekt benutzt kommen in diese Projekte der eigentliche Code und \keyword{absGDX-framework} wird als Dependency eingebunden. W�hrend wir jedoch das Framework noch entwickeln sind diese Projekte notwendig damit wir es auch ausprobieren k�nnen.

Da wir alle an dem Projekt mit der Eclipse IDE entwickeln haben wir ein paar extra Eclipse Einstellungen in die build.gradle Dateien ausgelagert:

\doinline
\begin{lstlisting}[caption=Eclipse Optionen in gradle setzen, title=\hspace{0 pt}, language=groovy]
eclipse.jdt.file.withProperties { props ->
    props.setProperty('org.eclipse.jdt.core.formatter.number_of_blank_lines_at_beginning_of_method_body', '0')
    props.setProperty('org.eclipse.jdt.core.formatter.number_of_empty_lines_to_preserve', '1')
    props.setProperty('org.eclipse.jdt.core.formatter.put_empty_statement_on_new_line', 'true')
    props.setProperty('org.eclipse.jdt.core.formatter.tabulation.char', 'tab')
    props.setProperty('org.eclipse.jdt.core.formatter.tabulation.size', '4')
    props.setProperty('org.eclipse.jdt.core.formatter.use_on_off_tags', 'false')
    props.setProperty('org.eclipse.jdt.core.formatter.use_tabs_only_for_leading_indentations', 'false')
    props.setProperty('org.eclipse.jdt.core.formatter.wrap_before_binary_operator', 'true')
    props.setProperty('org.eclipse.jdt.core.formatter.wrap_before_or_operator_multicatch', 'true')
    props.setProperty('org.eclipse.jdt.core.formatter.wrap_outer_expressions_when_nested', 'true')
    // ...
}
\end{lstlisting}

Dies ist ein gutes Beispiel wie m�chtig die Groovy Konfiguartionsdateien sind, wir f�gen in die generierten Eclipse property files hier noch unsere eigenen Felder ein. 
In diesem Beispiel setzen wir die Einstellungen f�r projektspezifische Formatierungen, damit der automatische Quellcodeformatter bei allen die das PRojekt in Eclipse laden gleich funktioniert.

\paragraph{Verwendung in unserem Projekt}

Um Problemen mit verschiedenen Versionen von gradle entgegen zu wirken ist in unserer git Repository nicht nur die build.gradle Dateien vorhanden sondern auch der \ti{gradle Wrapper}. Dies ist eine vollst�ndige unabh�ngige Version von gradle die man anstatt der lokal installierten verwenden soll. Damit ist garantiert dass jeder die gleiche Version von gradle verwendet.

Ausgef�hrt wird sie dann �ber die Kommandozeile mit Befehlen wie

\begin{lstlisting}[caption=Gradle in der Kommandozeile, title=\hspace{0 pt}, style=cmd]
> gradlew cleanEclipse eclipse afterclipseImport
\end{lstlisting}

Dieses Beispiel f�hrt zuerst den Task \ti{cleanEclispe} aus um alle Eclipse Dateien zu l�schen, dann werden sie mit \ti{eclipse} neu aus den gradle Einstellungen erzeugt und zu letzt werden mit \ti{afterEclipseImport} einige �nderungen vorgenommen. Der letzte Schritt von LibGDX vorgegeben damit das android Projekt richtig konfiguriert ist.