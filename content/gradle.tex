% Together with git
% Dependency Managment
% Build Managment
% Multi project files
% Maven Central Repository
% Groovy
% eclipse integration etc etc
% wrapper

Zwar commiten wir unser Code mit git in unsere Repository, jedoch nicht alle Dateien. Es gibt eine Reihe an Dateien die nur tempor�r sind und beispielsweise bei jedem Kompiliervorgang neu erzeugt werden. Es w�re sinnlos solche Dateien in die Versionskontrolle aufzunehmen, vor allem da sie sich immer �ndern w�rden. Es gibt jedoch auch Dateien die zwar wichtig sind, trotzdem aber nicht in die Versionskontrolle geh�ren. Dies sind einerseits Konfigurationsdateien mit lokalen Pfaden die ung�ltig auf anderen System w�ren, IDE spezifische Dateien die unn�tz f�r Nutzer anderer IDE's w�ren oder externe Libraries die oftmals zu gro� sind um sie alle in die Repository zu commiten.

Um diese Probleme k�mmert sich nun ein weiteres Tool namens gradle. Gradle ist ein Build-Managment-Automatisierung-Tool was bedeutet, dass wir in unser Git-Repository nur noch die gradle Konfigurationsdateien commiten und diese dann jeweils auf den Entwicklerrechnern ausf�hren. Diese l�st dann die Projektabh�ngigkeiten auf und l�dt fehlende Libraries automatisch nach und erstellt au�erdem Dateien wie die IDE-Projektdateien automatisch.

\paragraph{Dependency Managment}

In gradle kann man angeben von welchen Bibliotheken oder Projekten ein Programm abh�ngig ist. Diese Bibliotheken k�nnen dann wiederum von anderen Bibliotheken abh�ngig sein und so bildet sich ein Abh�ngigkeitsbaum ausgehen vom Anfangsprojekt.
Gradle l�dt dann diese Bibliotheken, falls sie noch nicht auf dem Rechner vorhanden sind. Dies hat auch den Vorteil dass wenn eine Bibliothek �fters im Abh�ngigkeitsbaum auftaucht sie trotzdem nur einmal geladen werden muss.

Gradle unterst�tzt viele verschiedene Repositories wo nach fehlenden Bibliotheken gesucht werden kann, standardm��ig wird jedoch die \ti{Maven Central Repository} benutzt.
In dieser Repository sind sehr viele der frei verf�gbaren java Libraries in vielen Versionen enthalten und k�nnen einfach �ber zum Beispiel gradle geladen werden.

\paragraph{Konfiguration}

Die Konfiguration geschieht �ber \keyword{build.gradle} Dateien. ..............