%TODO REFERENZEN !!!!!!!

Da wir an diesem Projekt zu zweit programmieren war es notwendig den Code zentral auf einem Server zu haben so dass wir beide Zugriff darauf haben.
Au�erdem brauchten wir ein System um Konflikte zu regeln falls wir zum Beispiel zeitgleich die gleiche Datei ver�ndern wollten.

Deshalb haben wir beschossen ein \ti{Revision Control System} zu verwenden. Genauer gesagt git. Mittels git ist der Code in einer Repository in der die �nderungen jeweils mit Commits eingepflegt werden. Auf Github haben wir ein Remote-Repository angelegt auf das wir beide nun unsere �nderungen pushen. Somit bekommt jeder die �nderungen des jeweils anderen mit. Falls ein Konflikt auftritt kann und muss dieser ebenfalls in git oder mit einem externen Tools gel�st werden bevor man die �nderungen pushen kann. Dieses System sorgt damit daf�r, dass man einfach zusammen an einem PRojekt arbeiten kann

\quickfigure{Git Network Graph mit smartgit}{git_network_graph}{0.7}

Au�erdem bietet git noch den Vorteil dass man �nderungen durch die einzelnen Commits zur�ckverfolgen kann und den Code unter Umst�nden auch auf einen alten Stand zur�cksetzen kann. Besonders beim Suchen nach Fehlern kann man so den Commit identifiziern bei dem dieser das erste mal aufgetreten ist. 

Git ist an sich ein dezentrales Protokoll und jeder Teilnehmer hat eine eigenst�ndige Kopie der Repository auf seinem Ger�t. Das bedeutet falls einer unserer Rechner ausf�llt - oder sogar der Server von Github, sind keine Daten verloren gegangen da der Code und die komplette History immer noch bei den �brigen Personen vorhanden ist. Dies ist jedoch kein Ersatz f�r ein Backup, da es trotzdem m�glich ist die Repository zu zerst�ren oder die History zu f�lschen.

Zwar kann man git direkt �ber eine Kommandozeile bedienen, jedoch ist es oft einfacher und �bersichtlicher daf�r spezielle Programme zu benutzen die einem �nderungen und die History visualisieren k�nnen. Wir benutzen daf�r das Programm smartgit, da dies f�r Open-Source Projekte kostenlos ist.

%TODO Smartgit screenshot ?
