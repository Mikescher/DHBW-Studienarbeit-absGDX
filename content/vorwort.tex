Eine gro�e Anzahl von Spielen, vor allem sogenannte \ti{mobile Games} teilen sich eine Reihe von Merkmalen.
So ist ein Gro�teil dieser Spiele zumindest von der Spiellogik her zweidimensional und die Karte ist in einem Grid angeordnet. Neben der gerasterten Karte gibt es eine Reihe von Entities die sich frei auf dieser bewegen k�nnen. Die Welt ist demnach entweder eine Draufsicht oder eine Seitenansicht. Im letzteren wird of einfache zweidimensionale Physik und Kollisionserkennung ben�tigt.

Da dies Merkmale sind, die diese Gruppe von Spielen gemeinsam haben, ist es unsere Idee ein Framework zu entwickeln dass diese Funktionalit�ten abstrahiert und es dem Entwickler somit einfacher macht Spiele dieser Art zu entwickeln.
Solche einfacheren 2D Spiele wie oben beschreiben findet man besonders h�ufig auf Handys und Tablets. Dies liegt einerseits an den schw�cheren Ger�ten, welche es nicht ohne weiteres schaffen grafisch aufwendigere 3D Umgebungen darzustellen und andererseits daran dass von solchen Gelegenheitsspielen einfache und schnell begreifbare Spielprinzipien erwartet werden.

Von technischer Seite haben wir uns dagegen entschieden ein Framework von Grund auf, ohne jede Basis zu schreiben. Denn in dieser Richtung gibt es schon mehrere fertige Produkte und es w�re unn�tig hier das Rad neu zu erfinden. 
Die Basis die wir f�r unser Projekt gew�hlt haben ist LibGDX. LibGDX ist OpenGL Framework mit dem Ziel Programme f�r unterschiedliche Plattformen zu kompilieren. Unterst�tzt werden unter anderen auch Android Ger�te und Desktop PCs. Dies bietet den Vorteil, dass man seine Programme auf dem Rechner schnell testen und debuggen kann und nur ab und zu auch auf ein echtes Android Ger�t exportieren muss. Haupts�chlich aus diesem Grund haben wir uns entschieden LibGDX als Grundlage f�r unser Framework zu verwenden

Das Ziel, dass wir uns f�r unser Produkt gesetzt haben ist es, es einfach zu gestalten 2D Spiele mit einer Tilemap und einem Entity-System zu entwickeln. Unser Framework �bernimmt die Verwaltung von Map, Entities und optional Dingen wie Men� und Netzwerk. Und durch LibGDX kann man die Projekte einfach direkt am Rechner testen.