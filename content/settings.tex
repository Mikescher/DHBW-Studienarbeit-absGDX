% => DependentProperties
% Tree structure

In absGDX gibt es zwei Arten von Einstellungen: Anwendungseinstellungen und Frameworkoptionen.
Anwendungseinstellungen sind Einstellungen im eigentlichen Spiel, ihre Verwaltung liegt in der Hand des Entwicklers.
Zus�tzlich gibt es verschiedene Optionen die angeben wie das Framework sich verh�lt. F�r beide Dinge werden \keyword{DependentProperty} Objekte verwendet.

\paragraph{Framework Optionen}

Eine Instanz der Klasse \keyword{DependentProperty} repr�sentiert einen einzelnen Wert. Dies ist meist ein Wahrheitswert kann aber auch ein String oder eine Zahl sein.
Der besondere Vorteil ist jedoch dass jede Property von einer anderen abh�ngig sein kann. Ein Tag ist nur dann aktiv wenn alle Tags von denen er abh�ngig ist ebenfalls aktiv sind. Daraus entsteht dann ein Abh�ngigkeitsbaum der einzelnen Properties.

\quickfigure{Die absGDX Einstellungen in Baumdarstellung}{settings_deptree}{0.7}

Dies ist beispielsweise f�r die Debugoptionen in absGDX n�tzlich: Wird der �bergeordnete Tag \ti{debugmode} deaktiviert sind auch alle anderen untergeordneten Debugoptionen deaktiviert. 
Ist die Property ein Boolean entscheidet sich ob er aktiv ist einfach mittels seines Wertes. 
Bei String-Properties werden alle Werte au�er \keyword{null} als wahr gewertet und Zahlen werden immer als wahr gewertet.

Zus�tzlich zu den normalen DependentProperties gibt es auch noch speziellere Subklassen:

\begin{itemize}
\item \keyword{ConstantBooleanProperty:} Dies ist eine BooleanProperty die nicht mehr ver�nderlich ist. Dies sit n�tzlich f�r zur Compile-zeit festgelegte Werte (wie ob debugging m�glich ist)
\item \keyword{MetaProperty:} Diese Property ist immer wahr und dient der Kategorisierung mehrere anderer Properties unter sich.
\item \keyword{RootProperty:} Dies ist �hnlich einer \keyword{MetaProperty} kann jedoch keine Abh�ngigkeiten haben und bildet somit die Wurzel des Baumes.
\end{itemize}

\paragraph{Laden und Speichern}

%TODO Loading and Saving