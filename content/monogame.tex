MonoGame ist ein Open-Source \CS Framework. Es unterstützt, ebenso wie Cocos2D-x, mehrere Zielplattformen. Neben Windows, Android und iOS kann man hier seine Programme aber auch auf der Playstation oder XBox ausführen.

MonoGame sieht sich selbst als Nachfolger des jetzt nicht mehr weiterentwickelten \cite{XNAEND} XNA Framework 4. Zu diesem Zweck wurde das Interface von Microsoft XNA 4 fast vollständig implementiert und in neueren Versionen auch erweitert.
Ein großer Vorteil von MonoGame ist .NET Framework. Microsoft stellt mit dem .NET 4.5 ein sehr mächtiges Framework zur Verfügung und mit MonoGame ist man in der Lage all seine Features zu nutzen.

MonoGame wäre unsere erste Wahl gewesen wenn es nicht einen starken Nachteil unter Android gäbe. Für Android gibt es keine kostenlose .NET VM und somit keinen kostenlosen Weg Programme die mit MonoGame geschrieben wurden auf Android auszuführen. Die einzigen bestehenden .NET Implementierungen unter Android sind kostenpflichtig und scheiden somit für uns aus.